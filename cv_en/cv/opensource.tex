%-------------------------------------------------------------------------------
%	SECTION TITLE
%-------------------------------------------------------------------------------
% \newpage
\cvsection{Open Source Contributions}

%-------------------------------------------------------------------------------
%	CONTENT
%-------------------------------------------------------------------------------
\begin{cventries}

%---------------------------------------------------------
\cvproject
{FedML-AI Community (contributor \& research intern)} % Project Title
{https://github.com/FedML-AI/FedML} % Project Link
{4k+} % Stars
{2022.06 - 2022.09} % Date(s)
{
\begin{cvitems} % Description(s) of tasks/responsibilities
\item {I enhanced \href{https://github.com/FedML-AI/FedCV}{\textbf{FedCV}} with the popular object detection model (e.g. YOLOv5, YOLOv7, YOLOv8, etc.), deployed them to produce environment and provided technical support for the community.}
\item {I completely ported the \href{https://github.com/owkin/FLamby}{\textbf{FLamby}} benchmark, a real-world medical dataset, to \href{https://open.fedml.ai/}{\textbf{FedML Open Platform}}.}
\end{cvitems}
} % Description


%---------------------------------------------------------

\cvproject
{hCaptcha-challenger (maintainer)} % Project Title
{https://github.com/QIN2DIM/hcaptcha-challenger} % Project Link
{1.3k+} % Stars
{2021.12 - 2023.10} % Date(s)
{
\begin{cvitems} % Description(s) of tasks/responsibilities
\item {We developed a robust AI-powered captcha solver utilizing Python and Selenium, effectively bypassing hCaptcha with an \textbf{accuracy exceeding 90\%}, and provided a user-friendly API for developers.}
\item {I employed the CLIP model to achieve zero-shot captcha image classification and clustering for automatically labeling the captcha images.}
\item {I released the \href{https://github.com/CaptchaAgent/hcaptcha-model-factory}{\textbf{hcaptcha-model-factory~(\faStar 66)}} with a comprehensive workflow for data collection, model training, and deployment for community.}
\end{cvitems}
} % Description


%---------------------------------------------------------

\cvproject
{AI-Paper-Collector (maintainer)} % Project Title
{https://github.com/MLNLP-World/AI-Paper-Collector}
{1.1k+} % Stars
{2021.12 - 2022.12} % Date(s)
{
\begin{cvitems} % Description(s) of tasks/responsibilities
\item {We designed and implemented an automated paper collector that efficiently retrieves over 10,000 research papers from top AI conferences (e.g., NeurIPS, ICLR, AAAI).}
\item {I built a user-friendly web interface allowing researchers to effortlessly search, filter, and download papers based on various criteria. This interface has seen 5,000+ unique users since its launch.}
\end{cvitems}
} % Description

%---------------------------------------------------------

\cvproject
{Awesome-FL (maintainer)} % Project Title
{https://github.com/youngfish42/Awesome-Federated-Learning-on-Graph-and-Tabular-Data}
{1.2k+} % Stars
{2023.06 - present} % Date(s)
{
\begin{cvitems} % Description(s) of tasks/responsibilities
\item {I actively contributed to the content curation, quality assurance, and maintenance of the \textbf{Awesome-FL} repository, a highly regarded resource for federated learning research. }
\end{cvitems}
} % Description


%---------------------------------------------------------

\cvproject
{Personal Projects} % Project Title
{https://github.com/beiyuouo} % Project Link
{} % Stars
{\href{https://github.com/beiyuouo}{\textcolor{text}{\faGithub~\textbf{\underline{beiyuouo}}}} (150+ followers, 490+ stars)} % Date(s)
{
\begin{cvitems} % Description(s) of tasks/responsibilities
\item {\href{https://github.com/beiyuouo/arxiv-daily}{\textbf{arxiv-daily~(\faStar 71)}}: Automatically collect and push the latest arXiv papers to GitHub using GitHub Actions.}
\item {\href{https://github.com/beiyuouo/awesome-asynchronous-federated-learning}{\textbf{awesome-asynchronous-federated-learning~(\faStar 70)}}: A collection of papers about asynchronous federated learning.}
\item {\href{https://github.com/beiyuouo/mid-air-draw}{\textbf{mid-air-draw~(\faStar 17)}}: A simple hand-drawn and gesture recognition system using YOLOv5.}
\end{cvitems}
} % Description


%---------------------------------------------------------
\end{cventries}
